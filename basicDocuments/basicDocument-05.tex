\documentclass[12pt]{scrreprt}
% changed from scrartcl

\usepackage[utf8]{inputenc}
\usepackage[T1]{fontenc}
\usepackage[ngerman]{babel}

\usepackage{graphicx}
%\includegraphics{KnuthAtOpenContentAlliance}

\author{Uwe Ziegenhagen}
\title{Meine erste Bachelor-Arbeit}
%\date{01.01.2017}

\usepackage{hyperref}
\hypersetup{
 bookmarks=true, % show bookmarks bar
 unicode=false,  % non - Latin characters in Acrobat’s bookmarks
 pdftoolbar=true,% show Acrobat’s toolbar
 pdfmenubar=true,% show Acrobat’s menu
 pdffitwindow=false, % window fit to page when opened
 pdfstartview={FitH},% fits the width of the page to the window
 pdftitle={My title},% title
 pdfauthor={Author}, % author
 pdfsubject={Subject},   % subject of the document
 pdfcreator={Creator},   % creator of the document
 pdfproducer={Producer}, % producer of the document
 pdfkeywords={keyword1} {key2} {key3},   % list of keywords
 pdfnewwindow=true,  % links in new window
 colorlinks=true,% false: boxed links; true: colored links
 linkcolor=blue,  % color of internal links
 filecolor=cyan, % color of file links
 citecolor=green, % color of file links
 urlcolor=magenta% color of external links
}


\begin{document}
\maketitle

\tableofcontents

\listoffigures

\listoftables

\chapter{Einleitung}

\section{Hallo}

\subsection{Froscon}

\subsubsection{jfoidsjfoijs}

\begin{figure}[h]
\begin{center}
\includegraphics[width=0.5\textwidth]{KnuthAtOpenContentAlliance}
\caption{Donald Knuth}\label{fig:don}
\end{center}
\end{figure}

Siehe Abbildung \ref{fig:don} auf Seite \pageref{fig:don}.

Auch gibt es niemanden, der den Schmerz an sich liebt, sucht oder wünscht, nur, weil er Schmerz ist, es sei denn, es kommt zu zufälligen Umständen, in denen Mühen und Schmerz ihm große Freude bereiten können. Um ein triviales Beispiel zu nehmen, wer von uns unterzieht sich je anstrengender körperlicher Betätigung, außer um Vorteile daraus zu ziehen?

Aber wer hat irgend ein Recht, einen Menschen zu tadeln, der die Entscheidung trifft, eine Freude zu genießen, die keine unangenehmen Folgen hat, oder einen, der Schmerz vermeidet, welcher keine daraus resultierende Freude nach sich zieht? Auch gibt es niemanden, der den Schmerz an sich liebt, sucht oder wünscht, nur, weil er Schmerz ist, es sei denn, es kommt zu zufälligen Umständen, in denen Mühen und Schmerz ihm große Freude bereiten können.

Um ein triviales Beispiel zu nehmen, wer von uns unterzieht sich je anstrengender körperlicher Betätigung, außer um Vorteile daraus zu ziehen? Aber wer hat irgend ein Recht, einen Menschen zu tadeln, der die Entscheidung trifft, eine Freude zu genießen, die keine unangenehmen Folgen hat, oder einen, der Schmerz vermeidet, welcher keine daraus resultierende Freude nach sich zieht? Auch gibt es niemanden, der den Schmerz an sich liebt, sucht oder wünscht, nur, weil er Schmerz ist, es sei denn, es kommt zu zufälligen Umständen, in denen Mühen und Schmerz ihm große Freude bereiten können. Um ein triviales Beispiel zu nehmen, wer von uns unterzieht sich je anstrengender körperlicher Betätigung, außer um Vorteile daraus zu ziehen?

\chapter{Hauptteil}

\section{Hallo2}

\subsection{Froscon2}

\subsubsection{jfoidsjfoijs2}

Auch gibt es niemanden, der den Schmerz an sich liebt, sucht oder wünscht, nur, weil er Schmerz ist, es sei denn, es kommt zu zufälligen Umständen, in denen Mühen und Schmerz ihm große Freude bereiten können. Um ein triviales Beispiel zu nehmen, wer von uns unterzieht sich je anstrengender körperlicher Betätigung, außer um Vorteile daraus zu ziehen?

Aber wer hat irgend ein Recht, einen Menschen zu tadeln, der die Entscheidung trifft, eine Freude zu genießen, die keine unangenehmen Folgen hat, oder einen, der Schmerz vermeidet, welcher keine daraus resultierende Freude nach sich zieht? Auch gibt es niemanden, der den Schmerz an sich liebt, sucht oder wünscht, nur, weil er Schmerz ist, es sei denn, es kommt zu zufälligen Umständen, in denen Mühen und Schmerz ihm große Freude bereiten können.

Um ein triviales Beispiel zu nehmen, wer von uns unterzieht sich je anstrengender körperlicher Betätigung, außer um Vorteile daraus zu ziehen? Aber wer hat irgend ein Recht, einen Menschen zu tadeln, der die Entscheidung trifft, eine Freude zu genießen, die keine unangenehmen Folgen hat, oder einen, der Schmerz vermeidet, welcher keine daraus resultierende Freude nach sich zieht? Auch gibt es niemanden, der den Schmerz an sich liebt, sucht oder wünscht, nur, weil er Schmerz ist, es sei denn, es kommt zu zufälligen Umständen, in denen Mühen und Schmerz ihm große Freude bereiten können. Um ein triviales Beispiel zu nehmen, wer von uns unterzieht sich je anstrengender körperlicher Betätigung, außer um Vorteile daraus zu ziehen?

\chapter{Fazit}

\section{Hallo3}

\subsection{Froscon3}

\subsubsection{jfoidsjfoijs3}

Auch gibt es niemanden, der den Schmerz an sich liebt, sucht oder wünscht, nur, weil er Schmerz ist, es sei denn, es kommt zu zufälligen Umständen, in denen Mühen und Schmerz ihm große Freude bereiten können. Um ein triviales Beispiel zu nehmen, wer von uns unterzieht sich je anstrengender körperlicher Betätigung, außer um Vorteile daraus zu ziehen?

Aber wer hat irgend ein Recht, einen Menschen zu tadeln, der die Entscheidung trifft, eine Freude zu genießen, die keine unangenehmen Folgen hat, oder einen, der Schmerz vermeidet, welcher keine daraus resultierende Freude nach sich zieht? Auch gibt es niemanden, der den Schmerz an sich liebt, sucht oder wünscht, nur, weil er Schmerz ist, es sei denn, es kommt zu zufälligen Umständen, in denen Mühen und Schmerz ihm große Freude bereiten können.

Um ein triviales Beispiel zu nehmen, wer von uns unterzieht sich je anstrengender körperlicher Betätigung, außer um Vorteile daraus zu ziehen? Aber wer hat irgend ein Recht, einen Menschen zu tadeln, der die Entscheidung trifft, eine Freude zu genießen, die keine unangenehmen Folgen hat, oder einen, der Schmerz vermeidet, welcher keine daraus resultierende Freude nach sich zieht? Auch gibt es niemanden, der den Schmerz an sich liebt, sucht oder wünscht, nur, weil er Schmerz ist, es sei denn, es kommt zu zufälligen Umständen, in denen Mühen und Schmerz ihm große Freude bereiten können. Um ein triviales Beispiel zu nehmen, wer von uns unterzieht sich je anstrengender körperlicher Betätigung, außer um Vorteile daraus zu ziehen?
\end{document}